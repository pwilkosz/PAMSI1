\documentclass[a4paper,11pt]{article}
\usepackage[polish]{babel}
\usepackage[OT4]{fontenc}
\usepackage[utf8]{inputenc}
\usepackage{graphicx}
\usepackage{subfig}
\usepackage{epstopdf}




\date{17/05/2014}


%opening
\title{PAMSI -- algorytm SIMPLEX}
\author{Piotr Wilkosz}
\captionsetup{belowskip=12pt,aboveskip=4pt}
\begin{document}

\maketitle

\section{Wstęp}
Celem ćwiczneia było zaprojektowanie algorytmu simplex. Algorytm ten ma za zadanie optymalizację tzw. \textit{funckji celu}, zależnej od zestawu 
zmiennych zwanych \textit{zmiennymi decyzyjnymi}. Na podstawie zadanych ograniczeń algorytm wyznacza w tym przypadku największą wartość funkcji.

\section{Zapis algorytmu}
Dane wejściowe algorytmu można zestawić w tabeli. Dane te są odpowiednio przetwarzane, następnie program wypisuje optymalne wartości zmiennych decyzyjnych
oraz wartość funkcji celu.

\section{Działanie}
Działanie algorytmu zostanie zaprezentowane na przykładzie problemu analizowanego podczas zajęć, jak również dla problemu nieco bardziej złożonego.
\section{Problem 1: Produkcja okien i drzwi}
W pewnej fabryce produkuje się drzwi i okna. Zysk z produkcji jednych drzwi wynosi 3000 zł, a z produkcji okna - 5000 zł. W pierwszym sektorze
fabryki produkuje się części do produkcji drzwi, wyprodukowanie komponentów dla jednej sztuki drzwi zajmuje godzinę. W drugim sektorze produkuje się części do produkcji okien, czas produkcji
komponentów dla jednego okna wynosi 2 godziny. W trzecim sektorze montowane są zarówno drzwi, jak i okna. Czas produkcji sztuki drzwi wynosi 3 godziny a produkcji okna - 2 godziny.
Pierwszy sektor może pracować 4 godziny w tygodniu, drugi 12 a trzeci 18 godzin w tygodniu.\newpage
Powyższe dane zapisać można w następujący sposób: \\
\begin{table}[h]
\centering
\caption{Dane wejściowe dla algorytmu simplex}
\begin{tabular}{|c|c|c|c|}
 \hline
  & czas prod. drzwi & czas prod. okna & dopuszczalny czas w tygodniu \\
  \hline
  sektor 1 & 1 & 0 & 4 \\
  \hline
  sektor 2 & 0 & 2 & 12\\
  \hline
  sektor 3 & 3 & 2 & 18\\
  \hline 
  zysk za sztuke & 3000 & 5000 & \\
  \hline
  \end{tabular}
\label{tab1}
\end{table}
Dane z tabeli~\ref{tab1} wprowadzono na wejście programu:

Rezultat działania programu:
\tiny
\begin{verbatim}
Podaj ilosc zmiennych decyzyjnych: 2
Podaj ilosc rownan: 3
wprowadz koszt dla zmiennej nr 1: 3000
wprowadz koszt dla zmiennej nr 2: 5000
Podaj kolejne wspolczynniki: 
pozycja: ( 1, 1 ): 1
pozycja: ( 1, 2 ): 0
pozycja: ( 2, 1 ): 0
pozycja: ( 2, 2 ): 2
pozycja: ( 3, 1 ): 3
pozycja: ( 3, 2 ): 2
Ograniczenie 1: 4
Ograniczenie 2: 12
Ograniczenie 3: 18
ROZWIAZANIE OPTYMALNE -> MAKSYMALNY ZYSK: 36000
WYPISANIE UKLADU: 
Z = 36000 * x0 + 0 * x1 + 0 * x2 + 0 * x3 + -1500 * x4 + -1000 * x5 + 
 X1 = 2 * X0 0 * X1 0 * X2 0 * X3 0.333333 * X4 -0.333333 * X5 
 X2 = 6 * X0 0 * X1 0 * X2 0 * X3 -0.5 * X4 0 * X5 
 X3 = 2 * X0 0 * X1 0 * X2 0 * X3 -0.333333 * X4 0.333333 * X5 
 X4 = 0 * X0 0 * X1 0 * X2 0 * X3 0 * X4 0 * X5 
 X5 = 0 * X0 0 * X1 0 * X2 0 * X3 0 * X4 0 * X5 
pwilkosz@pwilkosz:~/pamsi/lab1/prj$ 

\end{verbatim}
\normalsize
Z powyższego układu mozna odczytać $Z = 36000, x_{1} = 2, x_{2} = 6 $

\section{Problem 2: Maksymalizacja przychodu ze sprzedaży wyprodukowanych rur w osłonie plastikowej}
Pewna firma zajmuje się produkcją rur. Jednymi z produktów są rury sanitarne/grzewcze w osłonie wykonanej z tzw. rury korugowanej. Oba rodzaje rur
posiadają średnicę 16 lub 20 mm i umieszczane są w rurze korugowanej o średnicy odpowiednio 25 i 28 mm. W tabeli potrzebnej do wykonania algorytmu simplex
zostaną umieszczone dane, takie jak dobowa norma produkcji przedstawiona jako średni czas produkcji metra rury oraz cena za metr rury wraz z osłoną. Ponadto
z racji oczekiwań klienta ustalono maksymalny czas, w którym można produkować dany typ rury.

\begin{table}[h]
\centering
\caption{Dane wejściowe dla algorytmu simplex}
\begin{tabular}{|c|c|c|c|c|c|}
 \hline
  & rura 1 & rura 2 & rura 3 & rura 4 & dopuszczalny czas w ciągu doby [s] \\
  \hline
  prod 1 & 8,5 & 0 & 0 & 0 & 36000\\
  \hline
  prod 2 & 0 & 14,5 & 0 & 0 & 86400\\
  \hline
  prod 3  & 0 & 0 & 7,5 & 0 & 43200 \\
  \hline 
  prod 4  & 0 & 0 & 0 & 13 & 64800\\
  \hline 
  łączny czas prod.  & 8,5 & 14,5 & 7,5 & 13 & 86400\\
  \hline 
  zysk za sztuke & 5,83 & 7,57 & 5,80 & 8,2 & \\
  \hline
  \end{tabular}
\label{tab2}
\end{table}
\newpage
Rezultat działania programu:
\tiny
\begin{verbatim}
Podaj ilosc zmiennych decyzyjnych: 4
Podaj ilosc rownan: 5
wprowadz koszt dla zmiennej nr 1: 5.83
wprowadz koszt dla zmiennej nr 2: 7.57
wprowadz koszt dla zmiennej nr 3: 5.8
wprowadz koszt dla zmiennej nr 4: 8.2
Podaj kolejne wspolczynniki: 
pozycja: ( 1, 1 ): 8.5
pozycja: ( 1, 2 ): 0
pozycja: ( 1, 3 ): 0
pozycja: ( 1, 4 ): 0
pozycja: ( 2, 1 ): 0
pozycja: ( 2, 2 ): 14.5
pozycja: ( 2, 3 ): 0
pozycja: ( 2, 4 ): 0
pozycja: ( 3, 1 ): 0
pozycja: ( 3, 2 ): 0
pozycja: ( 3, 3 ): 7.5
pozycja: ( 3, 4 ): 0
pozycja: ( 4, 1 ): 0
pozycja: ( 4, 2 ): 0
pozycja: ( 4, 3 ): 0
pozycja: ( 4, 4 ): 13
pozycja: ( 5, 1 ): 8.5
pozycja: ( 5, 2 ): 14.5
pozycja: ( 5, 3 ): 7.5
pozycja: ( 5, 4 ): 13
Ograniczenie 1: 36000
Ograniczenie 2: 86400
Ograniczenie 3: 43200
Ograniczenie 4: 64800
Ograniczenie 5: 86400
ROZWIAZANIE OPTYMALNE -> MAKSYMALNY ZYSK: 62641.3
WYPISANIE UKLADU: 
Z = 62641.3 * x0 + 0 * x1 + -1.57615 * x2 + 0 * x3 + 0 * x4 + -0.0551131 * x5 + 0 * x6 + -0.142564 * x7 + 0 * x8 + -0.630769 * x9 + 
 X1 = 4235.29 * X0 0 * X1 0 * X2 0 * X3 0 * X4 -0.117647 * X5 0 * X6 0 * X7 0 * X8 0 * X9 
 X2 = 0 * X0 0 * X1 0 * X2 0 * X3 0 * X4 0 * X5 0 * X6 0 * X7 0 * X8 0 * X9 
 X3 = 5760 * X0 0 * X1 0 * X2 0 * X3 0 * X4 0 * X5 0 * X6 -0.133333 * X7 0 * X8 0 * X9 
 X4 = 553.846 * X0 0 * X1 -1.11538 * X2 0 * X3 0 * X4 0.0769231 * X5 0 * X6 0.0769231 * X7 0 * X8 -0.0769231 * X9 
 X5 = 0 * X0 0 * X1 0 * X2 0 * X3 0 * X4 0 * X5 0 * X6 0 * X7 0 * X8 0 * X9 
 X6 = 86400 * X0 0 * X1 -14.5 * X2 0 * X3 0 * X4 0 * X5 0 * X6 0 * X7 0 * X8 0 * X9 
 X7 = 0 * X0 0 * X1 0 * X2 0 * X3 0 * X4 0 * X5 0 * X6 0 * X7 0 * X8 0 * X9 
 X8 = 57600 * X0 0 * X1 14.5 * X2 0 * X3 0 * X4 -1 * X5 0 * X6 -1 * X7 0 * X8 1 * X9 
 X9 = 0 * X0 0 * X1 0 * X2 0 * X3 0 * X4 0 * X5 0 * X6 0 * X7 0 * X8 0 * X9 
pwilkosz@pwilkosz:~/pamsi/lab1/prj$ 


\end{verbatim}
\normalsize

Według obliczeń programu, największy dobowy przychód wynosi 62641 zł. Aby osiągnąć ten wynik, należałoby wyprodukować kolejne modele rury w następujących proporcjach:
$ x_{1} = 4235 m, x_{2} = 0 m, x_{3} = 5760 m, x_{4} = 554 m $

\section{Wnioski}

\begin{itemize}
 \item Algorytm simplex znajduje szerokie zastosowanie w wielu dziedzinach, począwszy od planowania produkcji, przez działania militarne, na serwisach randkowych kończąc.
 \item Algorytm pozwala na dokładne określenie poszczególnych zmiennych decyzyjnych (występują jedynie znikome błędy numeryczne).
 
\end{itemize}


\end{document}
